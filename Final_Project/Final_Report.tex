\documentclass[a4paper]{article}

\usepackage[a4paper,width=150mm,top=25mm,bottom=25mm]{geometry}
\usepackage[utf8x]{inputenc}
\usepackage{amsmath}
\usepackage{graphicx}
\usepackage[colorlinks]{hyperref}
\usepackage[table]{xcolor}

\title{Sentiment Analysis of Boğaziçi University Protests}
\author{Alper Yıldırım}

\begin{document}
\maketitle

\section{Introduction}


On 2 January 2021, Melih Bulu, a former deputy candidate of the ruling AKP in 2015, was appointed as the rector of Boğaziçi University by President Erdoğan. However, the rector of Boğaziçi University has been conventionally selected by an election between the faculty members, except the appointment in the 1980 coup d’état period. On January 4, hundreds of students started protests the assignment of Melih Bulu for the sake of democratic traditions and academic freedom. In a pretty short time, the police intervention had started, and the barricades in front of the campus were constructed. Later on, the clash between protestors and police has emerged, and police detained the students. \\

Nevertheless, the number of protestors significantly increased in a short span of time, the protests expanded to other cities such as Ankara or Izmir, the students in different universities started to participate in Boğaziçi University Protests, and each university’s students constituted their own “University Solidarity” for the organization of collective participation and support to the Boğaziçi University Protests. While the government authorities, including Minister of Interior Süleyman Soylu and President Erdoğan, pointed the protestors as a target, accused them of being terrorists, and even called LGBTQ+ people “perverts”. Under these repressive circumstances created by police violence, detentions, and government authorities’ discourse, the protestors used Twitter as a channel for communication, the spread of information, organization of protests, and reacting to the incidents during demonstrations. Therefore, Twittersphere offers insightful knowledge for Boğaziçi University Protests. In order to have better insights of the protestors' sentiments, I tried to perform a sentiment analysis of tweets regarding Boğaziçi University Protests in an unsupervised setting. \\ 

\section{Literature Review}

\subsection*{Theoretical Explanations for Social Movements}
	
Tüfekçi (2017), in her notable book 'Twitter and Tear Gas', gave point to the role of “digitally networked public sphere”. Her emphasis on digitally networked movements that are not “online-only” or “online-primarily” provides a forceful understanding of contemporary social movements (2017, p.6). Exemplifying Tahrir protests in 2011, Tüfekçi discussed the role of digital technologies on initial formation of social movements (2017). To put it in a different way, digital technologies, especially social media, can play a significant role for the organization of social movements. Moreover, Tüfekçi referred Morozov’s arguments on “slacktivism” which signifies the perplexing of people from productive activism and leading them to only clicking on political subjects online. Tüfekçi putted her counterargument that social media, and digital technologies in general, can play a recruiting role for activism, as empirical accounts supported it (2017, p. 16-17). On the other hand, Gerbaudo and Treré (2015) focused on the relationship between the construction of collective identity and social media activism such as the creation of a new iconography and lexicon for collective identities in the cases of Occupy wall Street or the Arab Spring (p.865-866). However, the authors emphasized that due to the dominance of empirical accounts, the investigation into collective identities in social media had been neglected. This is partly caused by, for Gerbaudo and Treré (2015), the ‘connective action’ theory of Bennett and Segerberg which claimed the new logic of social media movements does not require the requirement of collective identity framing because “the participation and engagement become highly flexible and personalized” (p.867). \\
	
On the other side of social movement literature, the dimension of emotions in social science research neglected for a long time, since human beings were historically evaluated as fully rational actors in scientific inquiry. Goodwin et al. (2001) explained the suspicious views on emotions in social science research with examples, stressing that the emotion-laden crowds in a social movement were seen as “psychologically ‘primitive’ group mind and group feelings” (2001, p.2). Freudian psychoanalysis, in particular, has a bad reputation on its evaluation of social movements, calling mobilized people “narcissistic, latently homosexual, oral dependent, or anal retentive” (2001, p.3). Goodwin and Jasper (2014) also underlined that most scholars evaluated social movements and protests as dangerous mobs, and the participants of social movements as “irrational, slaves of their emotions”. For Goodwin and Jasper, the analogy between the Nazis and the participants of the crowd in the 1950s was the last acceleration of the view of the crowds as blind and stupid (2014, p.5). However, in the following period, protestors were perceived either people have already his personal emotions or acquired them within the crowd (Goodwin et al., 2014). While the accounts considering emotions in social movement theories were developed in time, Goodwin et al. (2001) rightfully pointed that newspaper accounts are not helpful to identify emotions of protestors, since those newspaper accounts reflect the opinion of the reporters, not protestors. In that regard, Twitter, as a microblogging website, is a convenient medium to determine the emotions of protestors. At that point, the framing distinction developed by Snow and Benford (1988) came into prominence: diagnostic framing which is convincing potential protestors that a problem is required to be addressed in a movement, prognostic framing which is convincing protestors for suitable strategies and tactics, and lastly, motivational framing which is encouragement of protestors to barge in the protests (2001, p.6). Motivational framing is tightly related with the emotions of protestors, since Goodwin et al. (2001) properly propounded that “Cognitive agreement alone does not result in action” (p.6). Jasper (1997) asserted, as distinctly from political opportunity theorists, the Brown decision was an emotional stimulation for civil rights movement, rather than an objective indicator for the success of the movement. Moreover, Goodwin et al. (2001) stated that protest might be an expression way of one’s morals, pride or joy, along with the negative feelings such as guilt or shame (p.2). \\ 

At that point, returning Jasper (1998), the classification of emotions can be very useful for social movement research, especially in circumstances that the emotions of protestors are followed day by day. For instance, while anger is an emotion that has specific object and short-term, anxiety and joy are emotions that have not specific object, but they are also short-term; on the other hand, love and are long-term emotions that have specific objects, whereas shame and pride are long-term emotions but do not have specific objects (Goodwin et al., 2001, p.11). Furthermore, Goodwin et al. (2001) argued that there is a strong relationship between certain emotions and our understanding of incidents surrounding us, sometimes in an immedieate way rather than being an elaborative process. They claimed that “Protestors hope to create anger, […] choosing from among the available cultural repertory of label to identify their own emotions, presumably channeling them in the process” (Goodwin et al., 2001, p.14), and that statement considering negative sentiments is compatible with what I expect to find in the case of the Boğaziçi University Protests on Twitter. \\
	
Lastly, Jaspers (2008) studied social movements as moral protests from a cultural perspective, and distinguished pre-industrial and industrial forms of protest. Thereafter, he called “post-citizenship form of protest” as a third kind of social movement (Jaspers, 2008, p.7). To define this third kind of social movement, Jaspers claimed that people, who already satisfied with their own integration into political and economic systems of society, are seeking for the benefits for others such as environmentalist movements, animal rights movements, peace and disarmament movements, etc. In that regard, post-citizenship form of protest exceeds the concept of the basic rights and need for themselves. Further, Jaspers contrasted post-citizenship and citizenship movements, and asserted that post-citizenship movements “Unlike citizenship movements, […] flow easily into each other” (2008, p.7). Jaspers’ claims are substantial from two different viewpoints. Twitter activists can easily consist of protestors who do not pursue his or her own benefits. Additionally, Boğaziçi University Protests demonstrated the fluidity Jaspers mentioned, since it has started only a protest against the appointment of Melih Bulu but transformed also an LGBTQ+ movement. 
	
\subsection*{Computational Models of Sentiment Analysis}

Before moving on computational techniques of data collection and sentiment analysis, Hutter’s discussion on protest event analysis (PEA) can provide a plentiful groundwork for the research. As Hutter clearly defined it, protest event analysis (PEA) is a method of quantitative content analysis in which the amount and qualities of protests across diversified geographical regions and over time are systematically collected and analyzed (2014, p.335-336). However, for protest event analysis (PEA), the words in the sources such as newspapers or official reports should be quantified, otherwise, content analysis of textual sources would be a qualitative analysis. Referring Klandermans and Staggenborg (2002), Hutter stated that protest event analysis offers a method of measurement of political opportunities at cross-time and cross-regional levels, thus PEA is firmly related with political opportunities theory (2014). \\
	
In their study on random sampling in corpus design, Yörük et al. studied on the sampling methodology for the machine learning techniques of protest event collection to create a gold standard corpus (GSC). Even though keyword-based filtering for protest events coding is most common method, Yörük et al. asserted that their automated tool based on random sampling of protest events resulted with a better performance. As the authors referred, even though automated coding tools can deal with structural qualities of human language such as word counts, they face difficulty of extracting the meaning in the narrative content, since human language and knowledge are context dependent (p.4). Correspondingly, keyword-based models revealed weakness in such complex occurrences of indirect or less explicit expressions (p.7). In their study in the GLOCON Project, the “repertoire of contention” was defined in a wider sense, including strikes, rallies, boycotts, riots, clashes, and protests. Yörük et al. used both traditional and state-of-the-art machine learning model, namely BERT, and compared the results of their automated tool with four keyword-based publicly available tools: PolDem by Kriesi et al. (2019), MMAD by Weidman (2019), EventStatus Dataset by Huang et al. (2016), and the meta-analysis by Wang et al. (2016). The comparison demonstrated that the random training data with the use of BERT yields better performance than other models with keyword-filtered data. In conclusion, Yörük et al. offers a compromising automated tool of protest event collection. \\
    
In an early study of this research area, Pang et al. (2002) employed three particular machine learning methods for the sentiment classification problem: naïve Bayes, maximum entropy, and support vector machines (SVM). While their study focused only on positive and negative sentiments based on the classification of movie reviews, they revealed that the machine learning models performed better in comparison with humans’ performance. In their study, the SVM models performed best, and the Naïve Bayes performed worst by a narrow margin among their three supervised machine learning models. While Pang et al. (2002) tested only Feature Frequency and Feature Presence, O’Keefe and Koprinska (2009) studied the range of feature selectors and feature weights in order to achieve computationally less expensive models, focusing on naïve Bayes and support vector machine classifiers. \\
    
Gaind et al. (2019) developed an advanced model to extract emotions from social media texts based on six different Emotion-Categories. They combined two approaches. The first approach is built on Natural Language Processing (NLP) methods, whereas the second approach is built on machine learning models’ classification algorithms. For the machine learning part of their model, they have used SMO and J48 classifiers of the open-source library Weka, which classifiers returned high levels of accuracy: 91.7\% for SMO and 85.4\% for J48. Therefore, Gaind et al. (2019) achieved a promising model for the complex task of automatic emotion detection. Strapparava and Mihalcea (2008) also studied on the automatic detection and analysis of emotions, analyzed news headlines and constructed a dataset annotated for six emotions, which emotions are identical with Gaind et al. (2019) study. Their corpus consists of news titles, extracted from New York Times, CNN, and BBC News, and from the search engine of Google News. Strapparava and Mihalcea (2008) developed an unsupervised setting for their study, and evaluated their study through precision, recall, and F-measure levels which evaluation is similar to the study of Yörük et al. However, Strapparava and Mihalcea (2008) could not conclude a certain methodology that gives the most accurate outcome. That is to say, Gaind et al. (2019) study presented a more solid output tan Strapparava and Mihalcea (2008). \\
    
More recently, Sinpeng (2021) had examined the anti-government protests in Thailand in 2020 in which Twitter played a substantial role. Sinpeng argued that the role of Twitter in Thailand protests was not the mobilization of offline protest activities, rather construction of collective identities and the spread of information (2021). The Free Youth Movement (FYM) that Sinpeng focused is an organized group that started their online campaign on Twitter with \#FreeYouth hashtag. For the methodology, Sinpeng (2021) collected 27,233 tweets with the hashtag and used random sampling for the research. The content analysis of the sample demonstrated that only 4\% percent of tweets in hashtag was about the mobilization of the protestors, on the other hand, more than 70\% of tweets was about the grievance expression (Sinpeng, 2021). Further analysis exhibited that 66\% of tweets consists of the narrative about government opposition and democracy, whereas the narratives concerning youth rights and education correspond 29\% of the tweets. Lastly, even though Sinpeng (2021) found that the share of tweets on mobilization was pretty small, she underlined that Twitter had an important role for recruitment and mobilization of supporters at the initial phase of the movement, which result is compatible with the assertion of Tüfekçi (2017).

\section{Methodology}

I have compiled the tweets related with Boğaziçi University Protests using \verb|Twint| module. The tweets were collected using the common hashtags throughout the Boğaziçi University Protests. Selected hashtags for the data collection are:
\begin{itemize}
    \item \#BoğaziçiDireniyor
    \item \#BoğaziçiSusmayacak
    \item \#KabulEtmiyoruzVazgeçmiyoruz
    \item \#AşağıBakmayacağız
    \item \#LGBTİHaklarıİnsanHakları
    \item \#ArkadaşlarımızıSerbestBırakın
    \item \#BundanSonrasıHepimizde
\end{itemize}
Since the data collection was completed in early April, the collected corpus contains 221,740 tweets between the first day of the protests and the end of March. In order to achieve more accurate results, the duplicate tweets and non-Turkish tweets were excluded from the corpus. In order to achieve more accurate results, URLs, usernames, and hashtags were excluded from the dataframe using \verb|RegEx|. For the computational efficiency, all strings in the dataframe converted to lowercase characters. \\
	
Thereafter, the collected corpus converted to a bag-of-words vocabulary including both uni- and bigrams using \verb|CountVectorizer|. Nevertheless, the code performing Tf-Idf Vectorizer with uni- and bi-grams was also included in the files as comments. For the sentiment analysis, three different machine learning algorithms were selected: K-Means Clustering, Spectral Clustering, and Gaussian Mixture Model. Nevertheless, none of the models performed properly. The \verb|.score()| function of K-Means Clustering returned an irrelevant and rambling output: -6467830.1046. Spectral Clustering did not return any output, rather it returned an error message: "RuntimeError: nnz of the result is too large". Lastly, every time I run Gaussian Mixture Model, the computer system crashed. That is to say, Gaussian Mixture Model, similar to other two models, did not perform properly. \\

On the other hand, it is important to state that this methodology could contain selectivity bias. Some protestors might feel fear because of the risk of being judged for insult the President of Turkey, breach of the public peace, or aiding and abetting to terrorist organizations. Those Twitter protestors are likely to delete some of their tweets or protect their tweets in which case I cannot scrape their tweets because of ethical limitations. Furthermore, similar concerns of Twitter users created a Turkish Twittersphere in which there is a non-negligible number of anonymous users. Therefore, the prediction of demographical features of protestors such as gender, age, ethnicity has significant difficulties for Boğaziçi University Protests.
    
\section{Findings and Discussion}

Since none of the three models performed properly, I regretfully state that there is no findings in this research. The computational complexity of this task stood too far from my current abilities. One of the main difficulties was dealing one-dimensional structure of the data. Yet, to improve the model and achieve better results, the supervised settings can be used in a further study. Even though I tried to develop a supervised setting similar to the model of Hilal Benzer and Melike Ermiş \href{https://github.com/hilalbenzer/turkish-sentiment-analysis}{(Here is the link of GitHub repository)}, I have failed. Yet, there is still possibility to construct a supervised setting for the this task using K-Nearest Neighbors, Multinomial Naive Bayes, and Support Vector Machine algorithms. Lastly, rather than classifying only as negatives and positives, an emotion detection model, similar to Gaind et al. (2019) study, may offer plentiful insights for the social movements literature from the computational methodology, which should be considered for the further studies.
    
\section{References}

The references are alphabetically represented below: \\

Gaind, B., Syal, V., \& Padgalwar, S. (2019). Emotion detection and analysis on social media. arXiv preprint arXiv:1901.08458. \\

Gerbaudo, P., \& Treré, E. (2015). In search of the ‘we’ of social media activism: introduction to the special issue on social media and protest identities. \\

Goodwin, J., Jasper, J. M., \& Polletta, F. (2001). Introduction: Why emotions matter. Passionate politics: Emotions and social movements, 1. \\

Hutter, S. (2014). Protest event analysis and its offspring, in Donatella Della Porta (ed.), Methodological practices in social movement research, Oxford: Oxford University Press, pp. 335-367. \\

Jasper, J. (2008). The Art of Moral Protest. Chicago: University of Chicago Press. \\ https://doi.org/10.7208/9780226394961. \\

O’Keefe, T., \& Koprinska, I. (2009, December). Feature selection and weighting methods in sentiment analysis. In Proceedings of the 14th Australasian document computing symposium, Sydney (pp. 67-74). \\

Pang, B., Lee, L., \& Vaithyanathan, S. (2002). Thumbs up? Sentiment classification using machine learning techniques. arXiv preprint cs/0205070. \\

Sinpeng, A. (2021). Hashtag activism: social media and the \#FreeYouth protests in Thailand. Critical Asian Studies, 1-14. \\

Strapparava, C., \& Mihalcea, R. (2008, March). Learning to identify emotions in text. In Proceedings of the 2008 ACM symposium on Applied computing (pp. 1556-1560). \\

Tufekci, Z., “A Networked Public” in Twitter and Tear Gas: The Power and Fragility of Networked Protest (2017). Yale University Press. \\

Yörük E., Hürriyetoğlu A., Duruşan F., Yoltar Ç. Random Sampling in Corpus Design: Cross-Context Generalizability in Automated Multi-Country Protest Event Collection.

\end{document}